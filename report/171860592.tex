\documentclass[a4paper,UTF8]{article}
\usepackage{ctex}
\usepackage[margin=1.25in]{geometry}
\usepackage{graphicx}
\usepackage{amssymb}
\usepackage{amsmath}
\usepackage{amsthm}
\usepackage{tikz}
\usepackage{color}
\usepackage{listings}
\usepackage{cite}
% \usepackage[thmmarks, amsmath, thref]{ntheorem}
\theoremstyle{definition}
\newtheorem*{solution}{Solution}
\newtheorem*{prove}{Proof}
\usepackage{multirow}
\usepackage{url}
\usepackage[colorlinks,urlcolor=blue]{hyperref}
\usepackage{enumerate}


\renewcommand\refname{参考文献}
\setCJKmainfont{SimHei}

\setmonofont{Consolas}
\definecolor{mygreen}{rgb}{0,0.6,0}  
\definecolor{mygray}{rgb}{0.5,0.5,0.5}  
\definecolor{mymauve}{rgb}{0.58,0,0.82} 

\lstset{ %  
  backgroundcolor=\color{white},   % choose the background color; you must add \usepackage{color} or \usepackage{xcolor}  
  basicstyle=\ttfamily\footnotesize,
  % basicstyle=\footnotesize,        % the size of the fonts that are used for the code  
  breakatwhitespace=false,         % sets if automatic breaks should only happen at whitespace  
  breaklines=true,                 % sets automatic line breaking  
  captionpos=bl,                    % sets the caption-position to bottom  
  commentstyle=\color{mygreen},    % comment style  
  deletekeywords={...},            % if you want to delete keywords from the given language  
  escapeinside={\%*}{*)},          % if you want to add LaTeX within your code  
  extendedchars=true,              % lets you use non-ASCII characters; for 8-bits encodings only, does not work with UTF-8  
  frame=single,                    % adds a frame around the code  
  keepspaces=true,                 % keeps spaces in text, useful for keeping indentation of code (possibly needs columns=flexible)  
  keywordstyle=\color{blue},       % keyword style  
  %language=Python,                 % the language of the code  
  morekeywords={*,...},            % if you want to add more keywords to the set  
  numbers=left,                    % where to put the line-numbers; possible values are (none, left, right)  
  numbersep=12pt,                   % how far the line-numbers are from the code  
  numberstyle=\footnotesize\color{mygray}, % the style that is used for the line-numbers  
  rulecolor=\color{black},         % if not set, the frame-color may be changed on line-breaks within not-black text (e.g. comments (green here))  
  showspaces=false,                % show spaces everywhere adding particular underscores; it overrides 'showstringspaces'  
  showstringspaces=false,          % underline spaces within strings only  
  showtabs=false,                  % show tabs within strings adding particular underscores  
  stepnumber=1,                    % the step between two line-numbers. If it's 1, each line will be numbered  
  stringstyle=\color{orange},     % string literal style  
  tabsize=2,                       % sets default tabsize to 2 spaces  
  %title=myPython.py                   % show the filename of files included with \lstinputlisting; also try caption instead of title  
}  




%--

%--
\begin{document}
\title{\textbf{《计算机图形学》3月报告}}
\author{171860592,吕云哲,\href{mailto:maxwell.lyu@foxmail.com}{maxwell.lyu@foxmail.com}}
\maketitle

\section{综述}
\subsection{进度综述}
  本进度报告为持续更新版, 之后的进度会直接在该报告中添加, 直到实验正式结束, 将正式转为项目报告\\
  当前的进度如下: 
  \begin{itemize}
    \item cg\_algorithms.py: 完成全部功能
    \item cg\_cli.py: 完成全部功能
    \item cg\_gui.py: 完成全部必要功能, 可选功能持续添加中, 已经添加的额外功能
    \begin{itemize}
      \item [功能]将画布导出为PNG图片
      \item [功能]将画布导出为绘图命令, 兼容cg\_cli
      \item [功能]单行输入, 用cg\_cli命令绘图, 或将选中图元解析为命令
      \item [功能]操作历史记录, 撤销和重做
      \item [功能]控件选取颜色
      \item [交互]点击图元进行选中
      \item [交互]退出, 重置画布加入确认弹窗, 防止误操作
      \item [交互]根据选中的对象, 智能禁用不适用的操作, 例如椭圆不支持旋转
      \item [界面]优美界面, 设计风格参考Android 4.0 Holo Dark
      \item [界面]启动时显示Splash, 展现产品形象
      \item [界面]2种按钮样式, 4种按钮状态
    \end{itemize}
  \end{itemize}
  此后的部分, 已经撰写了项目综述和算法介绍, 而系统介绍暂时未撰写, 有待之后完成
\subsection{项目综述}
  本项目为2020年春季计算机图形学课程的课程设计, 以Python3为编写语言, 旨在通过实际的项目实践, 熟悉本课程教学内容, 掌握重点的图形绘制算法.\\
  项目的成果包括以下3个部分
  \begin{itemize}
    \item cg\_algorithms.py: 所有需要实现的算法
    \item cg\_cli.py: 基于算法, 构建命令行程序, 对算法进行应用于展示
    \item cg\_gui.py: 基于算法, 使用Pyqt5框架, 对算法进行应用与展示
  \end{itemize}
\section{算法介绍}
\subsection{DDA算法绘制线段}
\begin{itemize}
  \item 算法介绍\\
  DDA数值差分分析 (digital differential analyzer), 通过解直线的微分方程式, 进行直线的光栅化. 算法首先选取直线在两个方向的增量$\Delta x$和$\Delta y$中较大的一个, 作为一个光栅单位, 沿此方向扫描. 其利用了光栅扫描显示特性:像素列阵为屏幕单位网格, 通过离散取样, 使用x或y方向单位增量间隔, 逐步计算沿线路径各像素位置\\
  \item 算法流程
  \begin{enumerate}
    \item 首先计算x与y方向的增量, 此处与课堂教学内容不同, 可以避免由于直线垂直带来的麻烦
    \item 之后对直线退化为点的情况进行单独处理, 判断增量是否为0, 防止除零
    \item 分别计算两个方向的增量, 该步亦是对课堂教学内容的简化, 推导如下:\\
    当斜率$0\leq m\leq 1$
    \begin{align}
      \begin{split}
        length &= \max(\Delta x, \Delta y) = \Delta x\\
        x_{k+1} &= x_k + 1 = x_k + \frac{\Delta x}{\Delta x} = x_k + \frac{\Delta x}{length} = x_k + dx\\
        y_{k+1} &= y_k + m = y_k + \frac{\Delta y}{\Delta x} = y_k + \frac{\Delta y}{length} = y_k + dy
      \end{split}
    \end{align}
    当斜率$m > 1$
    \begin{align}
      \begin{split}
        length &= \max(\Delta x, \Delta y) = \Delta y\\
        x_{k+1} &= x_k + 1 = x_k + \frac{\Delta y}{\Delta y} = x_k + \frac{\Delta x}{length} = x_k + dx\\
        y_{k+1} &= y_k + \frac{1}{m} = y_k + \frac{\Delta x}{\Delta y} = y_k + \frac{\Delta y}{length} = y_k + dy
      \end{split}
    \end{align}
    因此不必费心处理沿x还是沿y递增, 直接使用下式进行迭代即可\cite{rog_2002}
    \begin{align}
      \begin{split}
        x_{k+1} &= x_k + dx\\
        y_{k+1} &= y_k + dy
      \end{split}
    \end{align}
    \item 不断递增, 并取整, 直接使用python自带的round函数进行
  \end{enumerate}
  
  

  \item 我的理解: DDA和最朴素的直线点斜式方程带入方法相比, 基本原理是一致的, 做出的改变在于节约了乘法, 其理解难度较低, 效率很高. 实际绘制时, 可以明显注意到效率高于Bresenham, 在后文绘制多边形时会详细讲解
  \item 相关代码
  \begin{lstlisting}[language={Python}]  
    elif algorithm == 'DDA':
        length = max(abs(y1 - y0), abs(x1 - x0))
        if length == 0: 
            result.append((x0, y0))
        else:
            dx = (x1 - x0) / length
            dy = (y1 - y0) / length
            i = 1
            x, y = x0, y0
            while i <= length:
                result.append((round(x), round(y)))
                x += dx
                y += dy
                i += 1\end{lstlisting}
\end{itemize}
\subsection{Bresenham算法绘制线段}
\begin{itemize}
  \item 算法介绍\\
  由Bresenham提出的算法, 采用整数增量运算, 精确而有效的光栅设备线生成算法, 也可用于其他曲线显示. 根据光栅显示原理, 离散取样过程中, 每一放样位置上只可能有2个像素接近于路径, 其核心部分为\begin{itemize}
    \item 设计整型参量, 用于判断候选像素与实际线路径之间的偏移关系
    \item 检测参量符号, 据此选取离路径近的像素
  \end{itemize}
  \item 算法流程\\
  课堂教学仅给出了第一八分之一圆域的绘制方法, 我参考《计算机图形学的算法基础》, 构造了任意方向的绘制算法
  \begin{enumerate}
    \item 准备工作, 计算$\Delta x$和$\Delta y$, 根据点的相对位置确认x与y在绘制时的增减情况, 并初始化x与y为绘制起点
    \item 判断斜率绝对值, 若超过1, 则交换dx和dy, 并在之后的流程中交换x与y的增减操作, 以下以第一八分之一圆域为例
    \item 初始化参量e为$2\Delta y-\Delta x$
    \item 重复$\Delta x$次\begin{enumerate}
      \item 绘制$(x,y)$
      \item 若e大于0, 说明绘制点$(x_{k+1}, y_{k+1})$, 此处使用s2, 可以同时解决负斜率和0斜率的问题, 因为python的sign函数对正数, 0, 负数分别返回1, 0, -1, 这步处理是很巧妙的. 更新e, 仅减去$2\Delta x$
      \item 若否, 说明绘制点$(x_{k+1}, y_{k+1})$, 该部分不改变y值和e值
      \item 更新x, 同样使用sign, 处理了无穷斜率的情况, 同时将e加上$2\Delta y$, 与前一步的$-2\Delta x$配合, 完成更新
    \end{enumerate}
    \item 最后, 需要将结束点加上
  \end{enumerate}
  \item 我的理解\begin{itemize}
    \item 算法本身有着很强的拓展性, 只要定义一个合适的决策参量, 该算法能够绘制很多其他的曲线
    \item 编写时, 我遇到的最大困难是耦合代码, 8个圆域, 还有水平垂直等特殊情况. 比较容易想到的方法, 是处理斜率超过1, 只需添加一个交换指示变量即可. 而使用sign函数将水平垂直和正负斜率均处理掉\cite{rog_2002}, 简直是绝妙的设计, 我十分佩服《计算机图形学的算法基础》一书的作者
  \end{itemize}
  \item 相关代码
  \begin{lstlisting}[language={Python}]     
    elif algorithm == 'Bresenham':
        x = x0
        y = y0
        dx = abs(x1 - x0)
        dy = abs(y1 - y0)
        s1 = sign(x1 - x0)
        s2 = sign(y1 - y0)
        interchange = (dy > dx)
        if interchange:
            dx, dy = dy, dx
        e = 2 * dy - dx
        for i in range(dx):
            result.append((x, y))
            if e > 0:
                if interchange:
                    x += s1
                else:
                    y += s2
                e = e - 2 * dx
            if interchange:
                y = y + s2
            else:
                x = x + s1
            e = e + 2 * dy
        result.append((x1, y1))\end{lstlisting}
\end{itemize}
\subsection{DDA算法绘制多边形/Bresenham算法绘制多边形}
\begin{itemize}
  \item 算法介绍\\
  绘制多边形, 其实就是绘制线段, 对于给定的n顶点列表, 按序连接顶点, 将得到n条线段. 用给定的算法对这些线段进行光栅化, 即得到绘制的多边形
  \item 算法流程
  \begin{enumerate}
    \item 处理2顶点的情况: 若顶点只有2个, 将不能构成有效的多边形, 此时按照线段处理, 用指定的算法绘制两点间的线段
    \item 对于至少3顶点的顶点列表, 利用Python列表循环访问的特性, 使用下标-1访问使用一个简单的for语句就能按序栅格化所有的线段
  \end{enumerate}
  \item 我的理解: 本次实验的多边形绘制相对简单, 因为实验要求并未规定是凸多边形, 因此我默认用户提供的顶点是按序合法的, 因此不进行凸包等运算, 按序连接即宣告完成
  \item 相关代码
  \begin{lstlisting}[language={Python}] 
    if len(p_list) == 2:
        return draw_line([p_list[0], p_list[1]], algorithm)
    result = []
    for i in range(len(p_list)):
        line = draw_line([p_list[i - 1], p_list[i]], algorithm)
        result += line
    return result\end{lstlisting}
\end{itemize}
\subsection{中点圆生成算法绘制椭圆}
\begin{itemize}
  \item 算法介绍\begin{itemize}
    \item 椭圆: 到2定点距离之和等于常数的点集
    \item 标准位置椭圆: 长轴和短轴平行于x轴和y轴
    \item 实验要求绘制的椭圆为标准位置椭圆, 并且不要求旋转, 因此\begin{itemize}
      \item 利用平移, 绘制以原点为中心的椭圆, 并将其平移到指定位置
      \item 利用对称性, 在平移的基础上, 只需绘制第一象限的1/4椭圆, 之后进行变换即可
    \end{itemize}
    \item 栅格化过程采取Bresenham算法的思路, 构造决策参数, 通过判断2个待选点的中点处, 决策函数的符号, 对下一位置的光栅化结果进行选取, 并迭代决策参数
    \item 另一个重点是, 椭圆在第一象限存在一点, 其斜率为-1. 以该点进行划分, 类似Bresenham算法绘制线段时遇到的, 斜率不同时分别沿x与y递增, 可以将椭圆分为两部分进行绘制\cite{sun_2006}. 计算该点的过程如下\begin{itemize}
      \item 当斜率为-1时, 有$$ -1$$
      \item 根据椭圆方程, 可构造$$df(x,y) = d(b^2x^2 + a^2y^2-a^2b^2) = 2b^2xdx + 2a^2ydy = d(0)  = 0$$
      \item 于是$$\frac{dy}{dx} = -\frac{b^2x}{a^2y}$$
      \item 故可以用$b^2x$与$a^2y$的相对大小作为条件, 区别区域1,2. 我的实现中, 斜率小于-1的区域标为区域1, 与课程PPT中的编号相反
    \end{itemize}
  \end{itemize}
  \item 算法流程
  \begin{enumerate}
    \item 框架代码指定参数为椭圆的矩形包围框左上角和右下角顶点坐标, 因此需要首先根据给定的参数, 计算出符合中点圆生成算法的参数, 包括椭圆的半长轴a, 半短轴b, 以及进行平移变换需要的中心点坐标$center = (x_c,y_c)$
    \item 参考\cite{rog_2002}, 首先计算一些辅助量, 以便加速之后的计算, 并使得算式更加简洁
    \item 设定初始点为$(a,0)$, 作为起始点, 并计算初始决策参数$$p1_0 = 2b^2x(x-1)+b^2/2+2a^2(1-b^2)$$
    \item 当$b^2x>a^2y$时, 为区域1, 逆时针移动, 不断进行如下迭代\begin{enumerate}
    \item 绘制点$(x,y)$
      \item 若$p1_k<0$, 选择像素$(x_k, y_{k+1})$, 并\begin{align}
        \begin{split}
          x_{k+1} &= x_k\\
          y_{k+1} &= y_k+1\\
          p1_{k+1} &= p1_{k} + 4a^2y + 2a^2\\
        \end{split}
      \end{align}
      \item 若$p1_k\geq 0$, 选择像素$(x_{k+1}, y_{k+1})$, 并\begin{align}
        \begin{split}
          x_{k+1} &= x_k - 1\\
          y_{k+1} &= y_k+1\\
          p1_{k+1} &= p1_{k} + 4a^2y + 2a^2 - 4b^2x\\
        \end{split}
      \end{align}
    \end{enumerate}
    \item 计算区域2的初始决策参数$$p2_0 = 2b^2(x^2+1)-4b^2x+2a^2(y^2+y-b^2) +a^2/2$$
    \item 当$b^2x\leq ^2y$且$x\geq 0$时, 为区域2, 逆时针移动, 不断进行如下迭代\begin{enumerate}
    \item 绘制点$(x,y)$
      \item 若$p2_k<0$, 选择像素$(x_k, y_{k+1})$, 并\begin{align}
        \begin{split}
          x_{k+1} &= x_k - 1\\
          y_{k+1} &= y_k+1\\
          p2_{k+1} &= p2_{k} - 4b^2x + 2b^2 + 4a^2y\\
        \end{split}
      \end{align}
      \item 若$p1_k\geq 0$, 选择像素$(x_{k+1}, y_{k+1})$, 并\begin{align}
        \begin{split}
          x_{k+1} &= x_k - 1\\
          y_{k+1} &= y_k\\
          p2_{k+1} &= p2_{k} - 4b^2x + 2b^2\\
        \end{split}
      \end{align}
    \end{enumerate}
    \item 之后首先沿x轴对称, 得到右半椭圆, 之后再沿y轴对称, 得到完整的椭圆, 最后使用之前求得的center坐标进行平移, 得到最终的绘制目标
  \end{enumerate}
  \item 我的理解\begin{enumerate}
    \item 优雅的代码: 使用lambda表达式, 1行一个变换, 利用python list的+=运算符, 简单地完成整个变换过程, 得到整个椭圆.
  \end{enumerate}
  \item 相关代码\begin{enumerate}
    \item 计算算法所需参数
    \begin{lstlisting}[language={Python}] 
      x0, y0 = p_list[0]
      x1, y1 = p_list[1]
      a = abs(x1 - x0) / 2
      b = abs(y1 - y0) / 2
      center = [round((x0 + x1)/2), round((y0 + y1)/2)]\end{lstlisting}
    \item 准备辅助量
    \begin{lstlisting}[language={Python}] 
      result = []
      x = round(a)
      y = round(0)
      taa = a * a
      t2aa = 2 * taa
      t4aa = 2 * t2aa
      tbb = b * b
      t2bb = 2 * tbb
      t4bb = 2 * t2bb
      t2abb = a * t2bb
      t2bbx = t2bb * x
      tx = x\end{lstlisting}
    \item 算法主体
    \begin{lstlisting}[language={Python}] 
      d1 = t2bbx *  (x-1) + tbb/2 + t2aa * (1-tbb)
      while t2bb * tx > t2aa * y:
          result.append((x, y))
          if d1 < 0:
              y = y + 1
              d1 = d1 + t4aa * y + t2aa
              tx = x - 1
          else:
              x =  x - 1
              y =  y + 1
              d1 = d1 - t4bb * x + t4aa * y + t2aa
              tx = x
      
      d2 = t2bb * (x*x +1) - t4bb*x+t2aa*(y*y+y-tbb) + taa/2
      while x>=0:
          result.append((x, y))
          if d2 < 0:
              x = x - 1
              y = y + 1
              d2 = d2 + t4aa * y - t4bb*x + t2bb
          else:
              x =  x - 1
              d2 = d2 - t4bb * x + t2bb\end{lstlisting}
    \item 对称与平移变换
    \begin{lstlisting}[language={Python}] 
      result += list(map(lambda x :(x[0],-x[1]), result))
      result += list(map(lambda x :(-x[0],x[1]), result))
      result = list(map(lambda x: (x[0] + center[0], x[1] + center[1]), result))\end{lstlisting}
  \end{enumerate}
\end{itemize}
\subsection{Bezier算法绘制曲线}
\begin{itemize}
  \item 算法介绍\\
  Bézier曲线是通过一组多边折线的各顶点唯一定义出来的。曲线的形状趋向于多边折线的形状,改变多边折线的顶点坐标位置和改变曲线的形状有紧密的联系。因此,多边折线有常称为特征多边形,其顶点称为控制顶点。一般,Bézier曲线段可拟合任何数目的控制顶点。Bézier曲线段逼近这些控制顶点,且它们的相关位置决定了Bézier多项式的次数。类似插值样条,Bézier曲线可以由给定边界条件、特征矩阵或混合函数决定,对一般Bézier曲线,方便的是混合函数形式。\\
  在我的实现中, 我并未对用户输入点的个数做限制, 也没有处理分段, 理论上可以绘制任意数量控制点的Bezier曲线. 实际使用中, 控制点达到6个时, 就能注意到明显的卡顿, 绘制曲线的过程等待超过1秒\\
  假设给出$n+1$个控制顶点位置$P_i=(x_i,y_i,z_i)$, 这些坐标点混合产生下列位置向量$P(u)$, 用于描述$P_0$和$P_n$间的逼近Bezier多项式函数的路径
  $$P_(u)=\sum_{i=0}^nP_iBEZ_{i,n}(u)(0\leq u\leq 1)$$
  利用 Bernstein 基函数的降(升)阶公式,得出 Bézier 曲线上点的坐标位置的有效方法 是使用递归计算。用递归计算定义的 Bézier 混合函数为\cite{sun_2006}
  $$BEZ_{i,n}(u)=(1-u)BEZ_{i, n-1}(u)+ uBEZ_{i-1,n-1}(u)$$
  这一函数的递归定义形式, 恰好可以用来构造递归函数的定义, 使用该递归函数即可对Bezier曲线进行光栅化\\
  另外一个需要考虑的问题, 是参量u的取值. 在该方程的定义中, 参量u可取连续值, 而实际的计算中, 参量u不可能取得连续的值, 因此需要确定一个nPoint, 即[0,1]区间中, u均匀取值的个数. 该step必须足够小, 使得绘制出来的曲线在显示上不断开. 为保证绘制效率, 该参量亦不能太小.
  \item 算法流程
  \begin{enumerate}
    \item 首先, 计算参量u使用的nPoint. 我的算法中, 该步骤是一个创新点:\begin{enumerate}
      \item 遍历p\_list控制点坐标列表, 对每一对相邻的控制点, 计算其切比雪夫距离(x坐标差值 和 y坐标差值 中的最大值). 
      \item 将算得的一系列距离相加.
      \item 正常绘制时, 取nPoint为距离和的2倍, 在一般情况下, 能够保证不出现曲线断开的情况
      \item 对于gui的变换部分, 由于涉及到快速的预览, 我加入了一个额外参数isQuick, 默认为False, 若为True, 则在距离和的基础上除以10, 作为参量u取值的个数, 以达到快速预览无卡顿的效果
    \end{enumerate}
    \item 使用上述nPoint, 用for循环在0~1间取参量u, 对每个参量u, 带入如下递归函数\begin{lstlisting}[language={Python}]
      def addPoint(p1, p2, u): 
          return ((1 - u) * p1[0] + u * p2[0], (1 - u) * p1[1] + u * p2[1])
      def bezier(p_list, r, i, u):
          if r == 0:
              return p_list[i]
          else:
              return addPoint(bezier(p_list, r - 1, i, u), bezier(p_list, r - 1, i + 1, u), u)
      pass    \end{lstlisting}
    该函数基本与递归计算的公式相同. 当r==0时, 说明递归结束, 直接返回i指定的控制点坐标, 否则使用点i与点i+1(递归过程中, i并非是控制点坐标, 而是$P_{i}^r$), 与参量u按照addPoint进行计算
    \item 之后, 使用lambda表达式完成舍入, 之后set()完成点的去重, 为绘图做优化, 最后构造list用作返回值
  \end{enumerate}
  \item 我的理解: Bezier曲线上的点, 实际上来自于控制点的线性和, 而每个控制点都有受u控制的权重. 对于一个确定次数的Bezier曲线, 实际上可以使用一个预先计算好的多项式代替递归过程, 以加速计算\\
  该加速策略在我的Bezier曲线中没有使用, 因为我的Bezier曲线没有限制控制点的个数, 因此这一加速策略仍然需要每次计算多项式, 并不会显著降低开销
  \item 相关代码如下
  \begin{lstlisting}[language={Python}]
    nPoints = 0
    for i in range(len(p_list) - 1):
        nPoints += max(abs(p_list[i][0] - p_list[i+1][0]), abs(p_list[i][1] - p_list[i+1][1]))
    if isQuick: nPoints = int(nPoints / 10)
    else: nPoints *= 2
    if algorithm == 'Bezier':
        n = p_list.__len__() - 1
        result = [p_list[0]]
        for i in range(1, nPoints):
            result.append(bezier(p_list, n, 0, float(i) / (nPoints - 1)))
        return list(set(map(lambda p: (round(p[0]), round(p[1])), result)))\end{lstlisting}
\end{itemize}
\subsection{3次均匀B样条算法绘制线段}
\begin{itemize}
  \item 算法介绍\\
  B样条曲线, 是Bézier曲线曲面的拓广. 相比于Bezier曲线, B样条对局部的控制能力强, 因此对实际出现的曲线的你和能力也较高. 但是同时, 也有着形状构造过程复杂, 初等曲线拟合能力差的缺点. \\
  本次实验中的3次均匀B样条, 与3次Bezier曲线比较相似, 使用如下递推公式进行
  $$N_{i,k}(u)=\frac{u-i}{k-1}N_{i,k-1}(u) + \frac{i+k-u}{k-1}N_{i+1,k-1}(u), k>1$$
  由于已经确定是3次, n=3, 利用均匀B样条曲线的平移性, 没有必要再使用递归方法, 而可以通过我在Bezier曲线时就设想的优化方案, 将多项式直接求出, 并硬编码在程序当中, 计算时直接带入u的值即可.\cite{cod_2006}\\
  $$\mathbf{S}_i(t) = \begin{bmatrix} \_t^3 & \_t^2 & \_t & 1 \end{bmatrix} \frac{1}{6} \begin{bmatrix} -1 &  3 & -3 & 1 \\  3 & -6 &  3 & 0 \\ -3 &  0 &  3 & 0 \\  1 &  4 &  1 & 0 \end{bmatrix} \begin{bmatrix} \mathbf{P}_{i-3} \\ \mathbf{P}_{i-2} \\ \mathbf{P}_{i-1} \\ \mathbf{P}_{i} \end{bmatrix}$$
  其中$3\leq i \leq n$, $t\in [0,1]$\\
  $i$的选取与$t$有关, 此处$t$在递推公式中为$u$, 需满足
  $$u_{i}\leq t <u_{i+1}$$
  而用于带入上述矩阵的$\_t$则是另外一回事, 其与$t$有如下转换
  $$\_t=\frac{t-u_i}{\Delta}=\frac{t-u_i}{u_{i+1}-u_i}$$
  于是, 只需首先构造$\{u_i\}$, 并使用Bezier曲线中类似的参量取值方法, 对每一个参量的取值$t$, 寻找其对应的$i$, 带入上述多项式中, 即可完成光栅化
  \item 算法流程
  \begin{enumerate}
    \item 起始部分与Bezier曲线的绘制共用, 选取参量, 这里不做介绍了
    \item 构造$\{u_i\}$列表, 形如公式6-231\cite{sun_2006}, 并归一化, 取值在0~1之间. 头尾的几个值其实并不会被使用, 其用途是与$i$的取值范围统一, 避免下标计算的麻烦
    \item 之后, 根据$t$的取值, 在$\{u_i\}$中查询, 找到合适的$i$
    \item 计算$\_t$, 使用$\_t$计算行向量$$\begin{bmatrix} \_t^3 & \_t^2 & \_t & 1 \end{bmatrix}$$与系数矩阵$$\frac{1}{6} \begin{bmatrix} -1 &  3 & -3 & 1 \\  3 & -6 &  3 & 0 \\ -3 &  0 &  3 & 0 \\  1 &  4 &  1 & 0 \end{bmatrix}$$的乘积$S$
    \item 最后计算$S_i(t)$, 即该参量$t$下, 绘制的点
  \end{enumerate}
  \item 我的理解: 3次均匀B样条有些类似3次Bezier曲线. 在Bezier曲线中, 一共只有4个控制点, 而在3次均匀B样条曲线中, 每4个连续的控制点都将使用由行向量$S$指定的权重, 计算出一些点, 而这些点拼接成了最终的b样条曲线\\
  这解释了B样条曲线局部控制能力强的原因, 因为每个控制点将会涉及到前后$3\Delta$的$u$范围内的点绘制, 而在Bezier中, 将影响整条曲线的绘制\\
  另外, B样条曲线至少需要4个控制点才能进行绘制
  \item 相关代码
  \begin{lstlisting}[language={Python}] 
    def bSpline(p, t):
    n = p.__len__() - 1
    m = 3 + n + 1
    step = 1 / (m - 2 * 3)
    u = [0, 0, 0] + [_u * step for _u in range(0, m - 2 * 3 + 1)] + [1, 1, 1]
    for i in range(3, m - 3):
        if u[i] <= t and t < u[i + 1]:
            break
    _t = (t - u[i]) / (u[i + 1] - u[i])
    A1 = (1 - _t) * (1 - _t) * (1 - _t) / 6.0
    A2 = (3 * _t * _t * _t - 6 * _t * _t + 4) / 6.0
    A3 = (-3 * _t * _t * _t + 3 * _t * _t + 3 * _t + 1) / 6.0
    A4 = _t * _t * _t / 6.0
    return (A1 * p[i-3][0] + A2 * p[i-2][0] + A3 * p[i-1][0] + A4 * p[i][0],
            A1 * p[i-3][1] + A2 * p[i-2][1] + A3 * p[i-1][1] + A4 * p[i][1])\end{lstlisting}
\begin{lstlisting}[language={Python}] 
  elif algorithm == 'B-spline':
      result = []
      for i in range(0, nPoints):
          t = i / nPoints
          result.append(bSpline(p_list, t))
      return list(set(map(lambda p: (round(p[0]), round(p[1])), result)))\end{lstlisting}
\end{itemize}
\subsection{平移变换}
\begin{itemize}
  \item 算法介绍\\
  平移变换实际上就是将原坐标$(x,y)$, 结合传入的参数$dx$, $dy$, 映射为$(x+dx, y+dy)$
  \item 算法流程\\
  一行lambda表达式搞定(也许能算是优雅的代码)\\
  使用Lambda表达式构造map, 也就是映射. 在Python3中, 无法直接使用一个lambda表达式, 从一个list构造其对应的list, 于是需要先构造map, 之后将其转换为list
  \item 相关代码
  \begin{lstlisting}[language={Python}] 
    return list(map(lambda p: (p[0] + dx, p[1] + dy), p_list))\end{lstlisting}
\end{itemize}
\subsection{旋转变换}
\begin{itemize}
  \item 算法介绍\\
  旋转变换也是线性变换, 根据PPT内容, 任意基准点$(x_r, y_r)$的旋转变换, 其结果为
  $$\begin{cases}
    x' &= x_r + (x-x_r)\cos\theta - (y-y_r)\sin\theta\\
    y' &= y_r + (x-x_r)\sin\theta + (y-y_r)\cos\theta
  \end{cases}$$
  \item 算法流程\begin{itemize}
    \item 由于要求的传入参数是角度, 需要首先计算出弧度制, 才能使用math库函数计算sin和cos
    \item 一行lambda表达式搞定
  \end{itemize}
  \item 相关代码
  \begin{lstlisting}[language={Python}]     
    rad = math.radians(r)
    sin, cos = math.sin(rad), math.cos(rad)
    return list(map(lambda p: (round(x + (p[0] - x)*cos -(p[1]-y)*sin), round(y + (p[0] - x)*sin +(p[1]-y)*cos)), p_list))\end{lstlisting}
\end{itemize}

\subsection{Cohen-Sutherland算法裁剪线段}
\begin{itemize}
  \item 算法介绍\\
  Cohen-Sutherland算法\cite{rog_2002}的核心, 是端点编码, 使用一个4位的编码描述线段的端点, 相对于给定的裁剪边界的位置信息:\begin{enumerate}
    \item 位1: 左边界以左取1, 否则取0
    \item 位2: 右边界以右取1, 否则取0
    \item 位3: 下边界以下取1, 否则取0
    \item 位4: 上边界以上取1, 否则取0
  \end{enumerate}
  获得位置信息之后, 通过两个端点的位置信息计算, 可得线段与窗口的位置关系\begin{enumerate}
    \item 两端点均为0000, 则线段完全在内部: 可以直接返回, 无需裁剪
    \item 两端点区域码逻辑与操作, 结果不为0000: 可以直接丢弃
  \end{enumerate}
  对于不满足这些特殊情况的点, 需要依次"左-右-上-下"与边界进行位置关系的判断. 当区域码相应的位为1, 则求交点, 用交点代替原来的点, 重新计算区域码, 直到剩下的线段完全在窗口内\\
  \item 算法流程与代码\begin{enumerate}
    \item 首先, 计算区域码, 采用位运算的方法进行
    \begin{lstlisting}[language={Python}] 
  def getPCode(x, y, window):
      pCodeList = []
      pCodeList.append(x < window[0])
      pCodeList.append(x > window[1])
      pCodeList.append(y < window[2])
      pCodeList.append(y > window[3])
      p = 0
      for i in pCodeList:
          p = (p << 1) | i
      return p\end{lstlisting}
    \item 判断线段与窗口的位置关系: 2: 完全在内; 1: 部分在内; 0: 完全在外\\
    \begin{lstlisting}[language={Python}] 
      def getFlag(pCode1, pCode2):
    return ((pCode1 == 0) & (pCode2 == 0)) + (pCode1 & pCode2 == 0)\end{lstlisting}
    这一步其实是用了一些骚操作, 接下来会解释骚操作是怎么work的\begin{itemize}
      \item +号前条件为区域码均为0000, +号后条件为区域码逻辑与为0
      \item 若二者都不成立, 因后者不成立, 已经可以判断完全在窗口外, 此时会返回0
      \item 若前者成立后者不成立? 这种情况是不可能的
      \item 后者成立前者不成立, 说明部分在内, 返回1
      \item 若皆成立, 因前者成立, 已经可以判断完全在内部, 此时会返回2
    \end{itemize}
    这牺牲了可读性, 实际上也并未有显著的效率提升, 但是我觉得很有意思, 就写了
    \item 若位置关系为2, 直接返回原list, 这里其实是拷贝了一份
    \begin{lstlisting}[language={Python}] 
      if vFlag == 2: 
          return [(x1, y1), (x2, y2)]
    \end{lstlisting}
    \item 若位置关系为1\begin{enumerate}
      \item 首先判断线段的指向, 水平(iFlag = 0), 水平(iFlag = 0), 垂直(iFlag = -1), 其余(iFlag = 1). 接下来对于每个方向, 都进行下面的操作
      \item 如果点1在内部(对于该方向), 先交换点1点2, 这样可以节约代码
      \item 如果不是垂直于x轴的线段, 且待判断的方向为左或右方向, 则使用该方向裁剪线段, 同时更新区域码
      \item 如果不是平行于x轴的线段, 且待判断的方向为上或下方向, 则使用该方向裁剪线段, 同时更新区域码
      \item 重新计算线段于窗口的位置关系, 若为1, 继续裁剪; 若为2, 返回当前的结果; 若为0, 返回空
    \end{enumerate}
    \begin{lstlisting}[language={Python}] 
  elif vFlag == 1:
      iFlag = 1
      if x1 == x2:
          iFlag = -1
      elif y1 == y2:
          iFlag = 0
      while(vFlag == 1):
          for i in range(1,5):
              if (pCode1 >> (4 - i)) != (pCode2 >> (4 - i)):
                  if (pCode1 >> (4 - i)) == 0:
                      x1, y1, x2, y2 = x2, y2, x1, y1
                      pCode1, pCode2 = pCode2, pCode1
                  if iFlag != -1 and i <= 2:
                      y1 = (y2 - y1) / (x2 - x1) * (window[i - 1] - x1) + y1
                      x1 = window[i - 1]
                      pCode1 = getPCode(x1, y1, window)
                  if iFlag != 0 and i > 2:
                      if iFlag != -1:
                          x1 = (x2 - x1) / (y2 - y1) * (window[i - 1] - y1) + x1
                      y1 = window[i - 1]
                      pCode1 = getPCode(x1, y1, window)
                  vFlag = getFlag(pCode1, pCode2)
                  if vFlag == 2:             
                      return [(round(x1), round(y1)), (round(x2), round(y2))]
                  elif vFlag == 0: 
                      return []\end{lstlisting}
    \item 若位置关系为0, 返回空
    \begin{lstlisting}[language={Python}] 
  else:
    return []\end{lstlisting}
    \end{enumerate}
  \item 我的理解: 该算法核心有2部分\begin{itemize}
    \item 区域码: 判断点与窗口的位置关系, 并基于此, 判断线段与窗口的位置关系
    \item 裁剪: 在各个方向裁剪求交点
  \end{itemize}
\end{itemize}

\subsection{Liang-Barsky算法裁剪线段}
\begin{itemize}
  \item 算法介绍\\
  Liang-Barsky算法\cite{rog_2002}, 其出发点是将2维裁剪问题化简为1维裁剪问题. \\
  该算法将带裁剪线段和裁剪窗口均看作一维点集, 其裁剪结果就是两个点集的交集.\\
  线段所在直线L与裁剪窗口的2交点为$Q_1Q_2$, 称之为诱导窗口, 而使用此诱导窗口对线段裁剪的结果, 与矩形窗口裁剪的结果是相同的\\
  该裁剪有参数化的裁剪条件\begin{align}
    \begin{split}
      x_{min} \leq x_1 + t\Delta x\leq x_{max}
      y_{min} \leq y_1 + t\Delta y\leq y_{max}
    \end{split}
  \end{align}
  参数化表示为$$t\times d_k\leq q_k$$
  定义参数$d, q$, 分别对应于左右上下边界, 有
  \begin{align}
    \begin{split}
      d_1 &= -\Delta x, \quad q_1 = x_1-x_{min}\\
      d_2 &= \Delta x, \quad q_2 = x_{max}-x_1\\
      d_3 &= -\Delta y, \quad q_3 = y_1-y_{min}\\
      d_4 &= \Delta y, \quad q_4 = y_{max}-y_1\\
    \end{split}
  \end{align}
  此处PPT中与教材\cite{sun_2006}的叙述不同, PPT与教材没有加符号, 我认为是不妥的\\
  以此作为条件, 有如下判断\begin{itemize}
    \item $d_k=0$, 线段平行于裁剪边界之一\begin{itemize}
      \item $q_k<0$, 说明完全在边界之外
      \item $q_k\geq 0$, 说明线段平行于裁剪边界, 且在窗口内
    \end{itemize}
    \item $d_k<0$, 线段从边界外部延伸到内部
    \item $d_k>0$, 线段从边界内部延伸到外部
  \end{itemize}
  \item 根据这些判断, 当$d_k\ne 0$时, 计算线段与边界$k$或其延长线交点的$t$值$$t=q_k/d_k$$
  \item 线段首先拥有$[t1,t2]$值区间[0,1], 表示不裁剪的整个线段, 之后的计算中, 使用如下规则, 使用每个交点的$t$值更新该区间\begin{itemize}
    \item $t_1>t_2$, 线段完全在窗口外, 抛弃之, 返回空
    \item $t_1\leq t_2$, 端点将确定裁剪的端点\begin{itemize}
      \item $t_1$值由线段从外而内遇到的矩形边界确定, $t_1$取0与各边界$t$值的最大值
      \item $t_2$值由线段从内而外遇到的矩形边界确定, $t_2$取1与各边界$t$值的最小值
    \end{itemize}
  \end{itemize}
  \item 完成整个过程后, 使用最终得到的$[t_1, t_2]$计算裁剪结果并返回
  \item 算法流程与代码\begin{enumerate}
    \item 算法中最重要的是函数clipt, 该函数使用传入的$d_k$与$q_k$值, 对线段在4个方向上进行裁剪, 并更新$t$取值区间
    \begin{lstlisting}[language={Python}] 
  def clipt(d, q, t_list): 
      visible = True
      if d == 0 and q < 0:
          visible = False
      elif d < 0:
          t = float(q) / d
          if t > t_list[1]:
              visible = False
          elif t > t_list[0]:
              t_list[0] = t
      elif d > 0:
          t = float(q) / d
          if t < t_list[0]:
              visible = False
          elif t < t_list[1]:
              t_list[1] = t
      return visible\end{lstlisting}\begin{enumerate}
      \item 首先初始化返回值为True, 表示线段课件, 存在一个部分位于裁剪窗口之内
      \item 如果$d=0$且$q<0$, 说明线段完全位于边界之外, 返回值设置为False
      \item 如果$d<0$, 由外向内延伸, 本次计算的$t$值将用来更新$t_1$值; 若更新后出现$t_1>t_2$, 设置返回值为False
      \item 如果$d>0$, 由内向外延伸, 本次计算的$t$值将用来更新$t_2$值; 若更新后出现$t_1>t_2$, 设置返回值为False
      \item 函数返回, 携带之前最后一次设置的返回值
    \end{enumerate}
    \item 主函数, 将依次在四个边界调用cilpt\begin{itemize}
      \item 若任意一次的返回值为False, 立刻结束, 并返回空
      \item 若均为True, 执行裁剪, 使用$[t_1,t_2]$区间, 并返回舍入后的2个裁剪后端点
    \end{itemize}
  \begin{lstlisting}[language={Python}] 
    elif algorithm == 'Liang-Barsky':
        t = [float(0), float(1)]
        deltax = float(x2 - x1)
        if clipt(-deltax, x1 - x_min, t):
            if clipt(deltax, x_max - x1, t):
                deltay = float(y2 - y1)
                if clipt(-deltay, y1 - y_min, t):
                    if clipt(deltay, y_max - y1, t):
                        if t[1] < 1:
                            x2 = x1 + t[1] * deltax
                            y2 = y1 + t[1] * deltay
                        if t[0] > 0:
                            x1 = x1 + t[0] * deltax
                            y1 = y1 + t[0] * deltay
                        return [(round(x1), round(y1)), (round(x2), round(y2))]\end{lstlisting}
  \end{enumerate}
  \item 我的理解: 该算法的思路是, 用线段所在直线与窗口, 计算线段在该窗口下的参数$t$区间, 各个边界算得到的区间求交集. 这一方法使得算法在每个边界至多进行一次求交点运算(实际上是除法运算), 效率较高
\end{itemize}

\section{系统介绍}
\dots

\section{总结}
\dots

\bibliographystyle{plain}%
%"xxx" should be your citing file's name.
\bibliography{myref}

\end{document}